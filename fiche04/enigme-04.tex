\documentclass[class=report,crop=false, 12pt]{standalone}
\usepackage[screen]{../myscratch}

\begin{document}


\titre[E]{Si ... alors ...}
%===============================


\begin{enigme}
\sauteligne

\begin{itemize}
  \item Scratch part de la position $x=-200$, $y=0$.
  \item Scratch s'oriente à $80$\textdegree\ (par rapport au Nord).
  \item Ensuite, il répète indéfiniment les instructions :
    \begin{itemize}
      \item avancer de 5,
      \item si $x > y$, alors afficher $x$ et s'arrêter.
    \end{itemize}
\end{itemize}

\bigskip

\textbf{Question.} Quelle est la première valeur de $x$ affichée ?
Arrondis la valeur de $x$ à l'entier inférieur.


%\begin{solution}
%$x=46,20$ donc la réponse est $46$ ou $47$.
%\end{solution}

\end{enigme}


\begin{enigme}


Scratch se déplace en fonction des touches de flèches pressées.
Il part de la position $x=0$, $y=0$ et est orienté vers la droite.

\begin{itemize}
  \item Si la flèche droite ($\rightarrow$) est pressée, alors Scratch avance de $30$
  (et attend $0.2$ seconde).
  \item Si la flèche haut ($\uparrow$) est pressée, alors Scratch tourne de $15$\textdegree\  vers la gauche (et attend $0.2$ seconde).  
\end{itemize}

Programme Scratch afin qu'il suive ces consignes.

\bigskip

Voici la séquence d'instructions saisie par un élève :

{\Large
$$\rightarrow \quad \rightarrow \quad \uparrow \quad \rightarrow \quad \uparrow \quad \rightarrow \quad \uparrow \quad \uparrow \quad \rightarrow \quad 
\rightarrow \quad \rightarrow \quad \uparrow \quad \rightarrow $$
}

\bigskip

\textbf{Question.} Quelle est la valeur de l'abscisse $x$ de la position de Scratch à la fin de ces instructions ?
Arrondis la réponse à l'entier inférieur.

%\begin{solution}
%$x=167,72$ donc la réponse est $167$ ou $168$.
%\end{solution}

\end{enigme}



\begin{enigme}

Scratch avance si certaines conditions sont validées.

\medskip

\begin{minipage}{0.49\textwidth}
\begin{itemize}
  \item Si \og{}$2<3$\fg{}, alors Scratch avance de $30$.
  
\vspace*{6ex}  
  
  \item Si \og{}$2+3=4$\fg{}, alors Scratch avance de $40$.
  
\vspace*{6ex}    
  
  \item Si \og{}$2 \times 3 > 7$ \textbf{ou} $9-5 > 3$\fg{}, alors Scratch avance de $50$. 
  
\vspace*{6ex}    
     
  \item Si \og{}\textbf{non} ($3 \times 4 = 12$)\fg{}, alors Scratch avance de $60$. 
\end{itemize}   
\end{minipage}
\begin{minipage}{0.49\textwidth}
\begin{center}
\setscratch{scale=0.6}

\begin{scratch}
  \blockif{si \booloperator{\ovalnum{2} < \ovalnum{3}} alors  }
  { 
    \blockmove{avancer de \ovalnum{30} pas}
  }

  \blockspace[0.5]

  \blockif{si \booloperator{\ovaloperator{\ovalnum{2} + \ovalnum{3}} = \ovalnum{4}} alors  }
  { 
    \blockmove{avancer de \ovalnum{40} pas}
  }

  \blockspace[0.5]

  \blockif{si \booloperator{
  \booloperator{\ovaloperator{\ovalnum{2} * \ovalnum{3}} > \ovalnum{7}}
  ou  
  \booloperator{\ovaloperator{\ovalnum{9} - \ovalnum{5}} > \ovalnum{3}}
  }
  alors  }
  { 
    \blockmove{avancer de \ovalnum{50} pas}
  }

  \blockspace[0.5]

  \blockif{si \booloperator{non \booloperator{\ovaloperator{\ovalnum{3} * \ovalnum{4}} = \ovalnum{12}}} alors  }
  { 
    \blockmove{avancer de \ovalnum{60} pas}
  }

\end{scratch}
\end{center}
\end{minipage}

%\medskip
%Quelles sont les conditions qui seront vérifiées ? Vérifie en programmant Scratch.

\bigskip

\textbf{Question.} Au total, après toutes ces instructions, de combien de pas Scratch a-t-il avancé ?


%\begin{solution}
%Les assertions 1 et 3 sont vraies, donc Scratch avance de $30+50=80$.
%\end{solution}

\end{enigme}


\end{document}

