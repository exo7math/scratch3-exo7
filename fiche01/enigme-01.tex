\documentclass[class=report,crop=false, 12pt]{standalone}
\usepackage[screen]{../myscratch}


\begin{document}


\titre[E]{Premiers pas}
%===============================


\begin{enigme}
Dans cette énigme, Scratch ne se déplace qu'horizontalement et verticalement. De plus, il ne peut avancer que de multiples de 50 pas : 50, 100, 150, 200...
Les instructions suivantes ont déjà été positionnées :
\begin{itemize}
  \item S'orienter à 90\textdegree\ (vers la droite)
  \item Avancer de 100
  \item S'orienter à 180\textdegree\  (vers le bas)
  \item Avancer de 100
  \item S'orienter à 90\textdegree\  (vers la droite)
  \item Avancer de 50  
\end{itemize}

\bigskip

\textbf{Question.} Aide Scratch à retourner à sa position de départ sans jamais passer deux fois au même endroit (c'est-à-dire sans couper son propre chemin). Combien de pas devrait-il faire, au minimum, pour retourner au départ ? 


%\begin{solution}
%Réponse : 350 pas.
%\end{solution}


\end{enigme}


\begin{enigme}

  Scratch a suivi le parcours dessiné ci-après.
  
Il se souvient de certaines dimensions (mesurées en pas), il y en a d'autres qu'il peut retrouver par le calcul, mais malheureusement il y a des dimensions dont il ne se souvient pas.

Programme ce parcours en choisissant des valeurs pour les dimensions inconnues.

\myfigure{1}{
\tikzinput{parcours1}
}

\bigskip

\textbf{Question.} Quelle est  la longueur (mesurée en pas) du parcours complet ?


%\begin{solution}
%Réponse : $480$ quelle que soit la longueur des segments inconnus.
%\end{solution}

\end{enigme}


\begin{enigme}
Trace des segments en suivant les instructions que voici :
\begin{itemize}
  \item \textbf{\'Etape 1.} Avancer de 50. Tourner de 10\textdegree.
  \item \textbf{\'Etape 2.} Avancer de 50. Tourner de 20\textdegree. 
  \item \textbf{\'Etape 3.} Avancer de 50. Tourner de 30\textdegree.   
  \item \ldots
  \item \ldots
\end{itemize}

\bigskip

\textbf{Question.} À quelle étape Scratch va-t-il recouper le parcours qu'il est en train de tracer ?


%\begin{solution}
%Réponse : à l'étape numéro 12.
%\end{solution}

\end{enigme}
\end{document}

