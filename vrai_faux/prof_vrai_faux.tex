\documentclass[class=report,crop=false, 12pt]{standalone}
\usepackage[screen]{../myscratch}

\begin{document}

\titre[P]{Vrai et faux}
%===============================



\section*{Objectifs}

\begin{itemize}
  \item Les booléens : vrai ou faux.
  \item Interviendront dans les conditions sur les tests et les boucles.
\end{itemize}


\section*{Durée}

2 à 3 heures (??)

\section*{Les activités}

\begin{itemize}
  \item Notion d'assertion : une phrase peut être vraie ou fausse selon la valeur de la variable $x$.
  \item Les portes logiques de base ET, OU, NON qui permettent de créer n'importe quel circuit logique.
  \item Une approche \emph{bit} à \emph{bit} de nombre binaire, par juxtaposition de $0$ et $1$. Notez que l'addition est ici une addition \emph{bit} à \emph{bit}, sans retenue (du type OU EXCLUSIF).
  \item On notera que pour les nombres binaires, addition et soustraction (sans retenue) c'est la même chose !
  \item Sur la partie code secret, c'est une première approche du chiffrement parfait (dans lequel on joute une clé de la longueur du message) et du chiffre de Vigenère (le message est découpé en bloc, une seule clé est ajoutée à chaque bloc).
  \item Le choix d'écrire des nombres binaires séparés par des points, par exemple $1.0.1.1$, permet de prendre conscience que c'est une écriture et évite l'écriture $1011$ qui prête à confusion. Par contre, cette notation n'est pas du tout standard.
\end{itemize}


\section*{Ressources}


\section*{People}

\begin{itemize}
  \item Auteur : Arnaud Bodin
\end{itemize}


\end{document}


