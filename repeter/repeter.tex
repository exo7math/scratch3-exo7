\documentclass[class=report,crop=false, 12pt]{standalone}
\usepackage[screen]{../myscratch}

\begin{document}

\titre[F]{Répéter}
%===============================

\begin{activite}
Une suite de couleurs est codée par ses initiales : \mot{R} pour rouge, \mot{V} pour vert, \mot{B} pour bleu. 
S’il y a 2 rouge à suivre on écrit \mot{2R} au lieu de \mot{R R}, s’il y a 3 bleu on note \mot{3B}.

Voici un exemple :

\myfigure{1}{
\tikzinput{repeter1}
}

Cette suite peut se coder \mot{R R R B V V B} ou plus simplement \mot{3R 1B 2V 1B}, pour 3 rouge, 1 bleu, 2 vert, 1 bleu.

\begin{enumerate}
  \item  Colorie les bulles en suivant le code :
  \begin{itemize}
  \item \mot{2B 2V 3R 1V 2B}

\myfigure{0.92}{
\tikzinput{repeter10}
}
  
  \item \mot{2V 4B 3V 1R}
  
\myfigure{0.92}{
\tikzinput{repeter10}
}

  \item \mot{5B 1V 4R}

\myfigure{0.92}{
\tikzinput{repeter10}
}
  
  \end{itemize}
  
  \item Trouve le code des suites de couleurs. Quand deux couleurs se suivent, utilise notre raccourci !

 \myfigure{0.91}{
\tikzinput{repeter2}
} 
   
\end{enumerate}

\end{activite}


\begin{activite}
Les directions sont codées suivant leur initiale : \mot{N} pour nord, \mot{S} pour sud, \mot{E} pour est et \mot{O} pour ouest. Si on fait deux pas de suite vers le nord, on écrit \mot{2N} au lieu de \mot{N N}. Si on fait cinq pas vers l’ouest, on écrit \mot{5O}.
\begin{enumerate}
  \item  Pars du point $A$ et avance suivant le code \mot{3E 1N 2O 2N 7E 2S}. Trace ton chemin. À quel point arrives-tu ?
  
\myfigure{0.84}{
\tikzinput{repeter4}
}  

  \item Repars du point $A$ avec le code \mot{1O 4N 6E 2N 2E 2S 2E 2S}. Trace ton chemin et indique sur quelle case tu arrives.

 
   
  \item  Écris le code du chemin allant du point $A'$ au point $B'$, puis celui du point $A'$ au point $C'$.


\myfigure{0.84}{
\tikzinput{repeter5}
}  

\end{enumerate}
\end{activite}


\begin{activite}
Un chemin est codé selon les instructions suivantes : \mot{1A} pour avancer d’un pas, \mot{2A} pour avancer de deux pas, \mot{3A} pour trois pas... \mot{G} indique un pivotement vers la gauche \emph{sans avancer} et \mot{D} indique un pivotement vers la droite \emph{sans avancer}. Par exemple, \mot{3A G 2A D 2A} indique que tu dois avancer de trois pas, pivoter sur la gauche, avancer de deux pas puis pivoter sur la droite et enfin avancer de deux pas.
\begin{enumerate}
  \item Pars du point $P$ en regardant dans la direction de la flèche et avance suivant les instructions \mot{3A G 1A D 2A D 2A G 3A}. Trace ton chemin. À quel point es-tu arrivé ?
  
\myfigure{0.86}{
\tikzinput{repeter6}
}   
  
  \item Repars du point $P$ avec les instructions \mot{1A D 3A G 2A G 1A D 2A D 1A}. Trace ton chemin et indique sur quelle case tu arrives.

  \item Écris le code d'un chemin qui part du point $P'$ et arrive au point $S'$ sans passer par les cases noires (plusieurs chemins sont possibles !). Est-il possible de trouver un chemin sans jamais tourner à droite ?

\myfigure{0.86}{
\tikzinput{repeter7}
} 
  
\end{enumerate}
\end{activite}


\begin{activite}
Une couleur est codée par son initiale : \mot{R} pour rouge, \mot{V} pour vert, \mot{B} pour bleu. 
Comme précédemment, s’il y a 2 rouge à suivre on écrit \mot{2R} au lieu de \mot{R R}, s’il y a 3 bleu on note \mot{3B}.
Voici un motif avec des répétitions : \mot{R V R V R V} que l’on code par \mot{3(R V)}, c’est-à-dire que l’on répète trois fois le motif \mot{R V}. Voici un autre motif avec des répétitions : \mot{2R B 2R B 2R B} que l’on code par \mot{3(2R B)}, c’est-à-dire que l’on répète trois fois le motif \mot{R R B}.

\begin{enumerate}

  \item Colorie les bulles en suivant le code :
  \begin{itemize}
    \item \mot{4(B V) 2B 5(R B)} 
  
\myfigure{0.92}{
\tikzinput{repeter20}
}    

    
    \item \mot{2R 3(B V) 2(B R V) 3(B R)}
    
\myfigure{0.92}{
\tikzinput{repeter20}
}    
 
    
    \item \mot{2(R V B) 3(2B V) \  B \ 2(V R)}

\myfigure{0.92}{
\tikzinput{repeter20}
}    

    
  \end{itemize}
    
  \item Trouve le code des suites de couleurs suivantes. Quand des motifs se répètent, utilise notre raccourci !

 \myfigure{0.91}{
\tikzinput{repeter3}
} 

\end{enumerate}

\end{activite}


\end{document}


