\documentclass[class=report,crop=false, 12pt]{standalone}
\usepackage[screen]{../scratch}

\begin{document}

\titre[P]{Boucles I}
%===============================



\section*{Objectifs}

\begin{itemize}
  \item Boucle "tant que".
  
  \item Lire un algorithme.
\end{itemize}


\section*{Durée}

2 heures (??)

\section*{Les activités}

\begin{itemize}
  \item Il est important de comprendre sur feuille ce qu'est une boucle. Il faut réfléchir à la structure avant de se jeter sur le clavier. Une boucle mal conçue conduit à un programme faux, voir à un boucle infinie.
  
  \item Bien comprendre le test qui donne ici la condition d'arrêt. Il existe aussi un type de boucle proche "répéter... jusqu'à ce que..."  qui exécute toujours au moins une fois le bloc d'instructions.
  
  \item Les diagrammes ont des limites : dès que l'on a un programme un peu sophistiqué, le diagramme devient vite illisible.
  
  \item Une autre activité est la lecture et l'analyse d'un algorithme. C'est d'ailleurs la situation la plus courante, vous avez un programme, vous savez à quoi il sert, mais il faut bien le comprendre, afin de le modifier par exemple.
\end{itemize}


\section*{Ressources}


\section*{People}

\begin{itemize}
  \item Auteur : Arnaud Bodin
\end{itemize}


\end{document}


