\documentclass[class=report,crop=false, 12pt]{standalone}
\usepackage[screen]{../myscratch}

\begin{document}

\titre[P]{Diviser pour régner}
%===============================



\section*{Objectifs}

\begin{itemize}
  \item Comprendre le concept "diviser pour régner".
\end{itemize}


\section*{Durée}

2 heures (??)

\section*{Les activités}

\begin{itemize}
  \item Le concept "diviser pour régner" n'est pas difficile à comprendre : il s'agit de séparer un problème compliqué en sous-problèmes plus simples. Ce qui est dur à comprendre, c'est que cela peut conduire à des algorithmes plus efficaces que ceux à quoi on pourrait s'attendre. 
  
  \item La première activité est une succession de petites devinettes assez classiques. Le principe est à peu près toujours dichotomique : on coupe en deux, puis on regarde d'un côté ou des deux côtés selon le problème.
  
  \item La seconde activité peut être l'occasion de tracer l'arbre de toutes les nombres : $32$ au sommet, avec deux fils $16$ et $48$, chacun ayant deux fils,...
  Ainsi les enfants au rang $n$ sont exactement ceux que l'on trouve en $n$ coups.
  
  \item On raconte que l'énigme de la troisième activité était proposée pour rentrer à Microsoft et que la réponse attendue était $19$. Pourtant on peut faire moins !!
  
  \item La découverte de la multiplication rapide par le jeune Karatsuba, alors âgé de 23 ans, fut un grand choc pour Kolmogorov, le père de la notion de complexité. Jusque là on pensait que l'on ne pouvait faire mieux que la multiplication habituelle. 
On n'aborde pas ici la notion de complexité, mais c'est une des notions fondamentales dans l'analyse d'un algorithme : L'algorithme est-il rapide ? Combien de mémoire utilise-t'il ?  
  Ce nouveau paradigme a ouvert la voie : on a depuis une autre façon de multiplier de grands entiers à l'aide de la transformée de Fourier rapide, qui est  encore meilleure que celle de Karatsuba.
\end{itemize}


\section*{Ressources}


\section*{People}

\begin{itemize}
  \item Auteur : Arnaud Bodin
  \item Idée après une discussion avec Martin Quinson sur son "crêpier psycho-rigide".
\end{itemize}


\end{document}


