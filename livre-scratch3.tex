\documentclass[11pt,twoside]{report}
%\documentclass[11pt,twoside,openright]{report}

\usepackage[screen]{myscratch}
%\usepackage[print]{myscratch}

\begin{document}

%%%%%%%%%%%%%%%%%%%%%%%%%%%%%%%%%%%%%%%%%%%%%%%%%%%%%%%%%%%%%%%%%%
% Titre + préface + sommaire
\renewcommand{\contentsname}{Sommaire}

% Préface
\import{divers/}{preface_livre_scratch.tex}

\debutchapitres


%%%%%%%%%%%%%%%%%%%%%%%%%%%%%%%%%%%%%%%%%%%%%%%%%%%%%%%%%%%%%%%%%%
% Les feuilles

\clearemptydoublepage
  
%======================================
\part{Activités Scratch}

%% 01 - Premiers pas
\import{fiche01/}{scratch-01.tex}

% 02 - Répéter
\import{fiche02/}{scratch-02.tex}

% 03 - Coordonnées
\import{fiche03/}{scratch-03.tex}

% 04 - Si ... alors ...
\import{fiche04/}{scratch-04.tex}

% 05 - Entrée/Sortie
\import{fiche05/}{scratch-05.tex}

% 06 - Variables et hasard
\import{fiche06/}{scratch-06.tex}

% 07 - Si ... alors ... sinon ...
\import{fiche07/}{scratch-07.tex}

% 08 - Plusieurs lutins
\import{fiche08/}{scratch-08.tex}

% 09 - Sons
\import{fiche09/}{scratch-09.tex}

% 10 - Invasion
\import{fiche10/}{scratch-10.tex}

% 11 - Créer ses blocs
\import{fiche11/}{scratch-11.tex}

% 12 - Listes
\import{fiche12/}{scratch-12.tex}


\clearemptydoublepage
\setcounter{chapter}{0}

%======================================
\part{Énigmes Scratch}

%% 01 - Premiers pas
\import{fiche01/}{enigme-01.tex}

% 02 - Répéter
\import{fiche02/}{enigme-02.tex}

% 03 - Coordonnées
\import{fiche03/}{enigme-03.tex}

% 04 - Si ... alors ...
\import{fiche04/}{enigme-04.tex}

% 05 - Entrée/Sortie
\import{fiche05/}{enigme-05.tex}

% 06 - Variables et hasard
\import{fiche06/}{enigme-06.tex}

% 07 - Si ... alors ... sinon ...
\import{fiche07/}{enigme-07.tex}

% 08 - Plusieurs lutins
\import{fiche08/}{enigme-08.tex}

% 09 - Sons
\import{fiche09/}{enigme-09.tex}

% 10 - Invasion
\import{fiche10/}{enigme-10.tex}

% 11 - Créer ses blocs
\import{fiche11/}{enigme-11.tex}

% 12 - Listes
\import{fiche12/}{enigme-12.tex}

% Solutions des énigmes
\import{solutions/}{solutions_scratch.tex}


\clearemptydoublepage
\setcounter{chapter}{0}

%======================================
\part{Activités sur feuilles}

\import{premiers_pas/}{premiers_pas.tex}

\import{repeter/}{repeter.tex}

\import{operations_algebriques_1/}{operations_algebriques_1.tex}

\import{vrai_faux/}{vrai_faux.tex}

\import{operations_algebriques_2/}{operations_algebriques_2.tex}

\import{si_alors/}{si_alors.tex} 

\import{boucles_1/}{boucles_1.tex} 

\import{chercher/}{chercher.tex}

\import{puissances_de_2/}{puissances_de_2.tex} 

\import{binaire/}{binaire.tex}

\import{boucles_2/}{boucles_2.tex} 

\import{graphe/}{graphe.tex}

\import{base_de_donnees/}{base_de_donnees.tex}

\import{pixels/}{pixels.tex}

\import{diviser_pour_regner/}{diviser_pour_regner.tex}

\import{couleurs/}{couleurs.tex}

\import{crypto/}{crypto.tex}

\import{triangulation/}{triangulation.tex}

\import{distance_mots/}{distance_mots.tex}

\clearemptydoublepage
\setcounter{chapter}{0}

%======================================
\part{Énigmes sur feuilles}

% Enigmes
\import{enigmes_feuilles/}{enigmes_feuilles.tex}

% Solutions des énigmes
\import{solutions/}{solutions_feuilles.tex}


%%%%%%%%%%%%%%%%%%%%%%%%%%%%%%%%%%%%%%%%%%%%%%%%%%%%%%%%%%%%%%%%%%
% Postfac

\import{divers/}{postface_livre_scratch.tex}

\vfill
\bigskip
\bigskip

\centerline{Version 2.00 -- Mai 2021}



\end{document}

