\documentclass[class=report,crop=false, 12pt]{standalone}
\usepackage[screen]{../myscratch}

\begin{document}

%\newcommand{\hexa}{\text{hex}}


\titre[E]{Solutions -- \'Enigmes sur feuilles}
%===============================



%------------------------------------
%\section*{Feuilles 1-2-3}

\begin{enigme}[Premiers pas]

Réponse : la case B1 (en partant de D3).
\myfigure{0.8}{
  \tikzinput{sol_premiers_pas}
}  

\end{enigme}




\begin{enigme}[Répéter]

Réponse : \mot{RBBV}.

\mot{X} = \mot{V} \quad \mot{Y} = \mot{B} \quad \mot{Z} = \mot{R}

\myfigure{0.5}{
  \tikzinput{sol_repeter}
}  

\end{enigme}



\begin{enigme}[Opérations algébriques I]

Réponse : $x=9$.

Vérifions que $x=9$ est la solution.
\begin{itemize}
  \item $x \leftarrow x-3$, maintenant $x$ vaut $9-3$, c'est-à-dire $6$.
  \item $x \leftarrow x \times x$, maintenant $x$ vaut $6 \times 6$, c'est-à-dire $36$.
  \item $x \leftarrow x - 27$, et enfin $x$ vaut $36-27$, c'est-à-dire $9$ !
\end{itemize}
    
\end{enigme}



%------------------------------------
%\section*{Feuilles 4-5-6}


\begin{enigme}[Vrai et faux]

Réponse : $14$ (lampes $1$ et $4$). 

\end{enigme}


\begin{enigme}[Opérations algébriques II]

Réponse : $2$.

\begin{itemize}
  \item $x_1 = 31,500 = 3,1500e1$
  \item $nf(a)=142,88 = 1,4288e2$
  \item $nf(b) = 142,87 = 1,4287e2$
  \item $nf(nf(a)-nf(b)) = 0,0300 = 3e-2$
  \item $x_2 = nf(\frac{1}{0,0300}) = 33,333 = 3,3333e1$
  \item $x_2-x_1 = 33,333-31,500 = 1,8333 \simeq 2$ 
\end{itemize}


\end{enigme}


\begin{enigme}[Si ... alors ...]
Réponse : $n=7$.
\end{enigme}


%------------------------------------
%\section*{Feuilles 7-8-9}


\begin{enigme}[Boucles I]

Réponse : $70$.

\end{enigme}



\begin{enigme}[Chercher et remplacer]

Réponse : 6. Liste : \mot{crise}, \mot{miracle}, \mot{miette}, \mot{casser}, \mot{rail}, \mot{crasse}.

\end{enigme}


\begin{enigme}[Puissances de 2]

Réponse : $\frac{1}{100} \times 2^{20} = 10485,76$ cm, la réponse est donc $10\, 486$.%$10485$ ou $10486$.

\end{enigme}


%------------------------------------
%\section*{Feuilles 10-11-12}


\begin{enigme}[Binaire]

Réponse : $55$, car $a = 1101100$, $b = 1011001$,
$a \otimes b = 1001000$, $\text{NON}(a\otimes b) = 0110111 = 55$.

\end{enigme}



\begin{enigme}[Boucles II]

Réponse : 263514.

\begin{center}
\begin{minipage}{0.8\textwidth}
\textbf{2.}\indentation Tant que $n\ge4$:\\
\textbf{6.}\indentation\indentation $n \leftarrow n - 4$\\
\textbf{3.}\indentation Si (($n=1$) ou  ($n=3$)):\\
\textbf{5.}\indentation\indentation Afficher "Ce nombre est impair."\\
\textbf{1.}\indentation Sinon:\\
\textbf{4.}\indentation\indentation Afficher "Ce nombre est pair."\\
\end{minipage}
\end{center} 


\end{enigme}


\begin{enigme}[Graphe]


Réponse : RVRR.

\myfigure{0.9}{
  \tikzinput{sol_graphe}
}  


\end{enigme}


%------------------------------------
%\section*{Feuilles 13-14-15-16}


\begin{enigme}[Bases de données]

Réponse : 1250 (lignes 1, 2, 5, 6 et 9 de la table 4).

\end{enigme}




\begin{enigme}[Pixels]

Réponse : 4

\myfigure{0.7}{
  \tikzinput{sol_pixels}
} 

\end{enigme}




\begin{enigme}[Diviser pour régner]

Réponse : $169$. En effet, les gaulois \textbf{0}, \textbf{3}, \textbf{5}, \textbf{7} n'ont pas de pouvoirs magiques, donc la bouteille porte le numéro
$n = 2^\mathbf{0}+2^\mathbf{3}+2^\mathbf{5}+2^\mathbf{7} = 169$.

\end{enigme}



\begin{enigme}[Couleurs]

\newcommand{\hexa}{\text{hex}}

Réponse : $B7_\hexa = 183$. ($R = 206$, $V = 182$, $B = 125$, $G = 183.05$, $G' = 183 = B7_\hexa$)

\end{enigme}


%------------------------------------
%\section*{Feuilles 17-18-19}


\begin{enigme}[Cryptographie]

\definecolor{coul_prive}{rgb}{0.93,0.26,0}
\definecolor{coul_public}{rgb}{0.06,0.63,0}

\newcommand{\prive}[1]{\relax\ifmmode{\color{coul_prive} #1}\else{\bf\color{coul_prive} #1}\fi}
\newcommand{\public}[1]{\relax\ifmmode{\color{coul_public} #1}\else{\bf\color{coul_public} #1}\fi}


Réponse : $11$.

Clé = $(4, 11, 7)$

\prive{A VAINCRE SANS PERIL ON TRIOMPHE SANS GLOIRE}

\end{enigme}



\begin{enigme}[Triangulation]

Réponse : 17 ($4$ arêtes partent de $A$, $7$ partent de $J$ et $6$ partent de $M$).

\myfigure{1.2}{
  \tikzinput{sol_triangulation}  
}

\end{enigme}


\begin{enigme}[Distance entre deux mots]
Réponse : 6.

\myfigure{0.8}{
  \tikzinput{sol_distance}  
}

\end{enigme}

\end{document}


