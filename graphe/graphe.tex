\documentclass[class=report,crop=false, 12pt]{standalone}
\usepackage[screen]{../myscratch}

\begin{document}

\titre[F]{Graphes}
%===============================


Un \emph{graphe} du plan est un ensemble de points, appelés \emph{sommets}, reliés par des lignes, appelées \emph{arêtes}.

À gauche, voici un exemple de graphe, ayant $5$ sommets et $8$ arêtes (il y a un arête qui relie un sommet à lui-même). Attention : dans cette fiche deux arêtes n'ont pas le droit de se couper (voir les figures de droite).


\myfigure{1}{
  \tikzinput{graphe-ex1-1}
} 





\begin{activite}[Le théorème des 4 couleurs]

À une carte géographique, on associe un graphe de la façon suivante :
  \begin{itemize}
    \item pour chaque pays on crée un sommet en choisissant un point à l'intérieur du pays (par exemple la capitale) ;
    \item on relie deux sommets si les deux pays associés ont une frontière commune.
  \end{itemize}
  
  
  Voici un exemple avec une partie de l'Europe.
  Il y a un sommet pour la France ($F$), un pour l'Espagne ($E$), un pour le Portugal ($P$)...
  Le sommet $F$ est relié au sommet $E$ car la France et l'Espagne ont une frontière commune, le sommet $E$ est relié au sommet $P$ car l'Espagne et le Portugal ont une frontière commune, par contre les sommets $F$ et $P$ ne sont pas reliés car il n'y a pas de frontière entre la France et le Portugal.
  
 \myfigure{1}{
  \tikzinput{graphe-ex1-2}
} 
  
  Voici un autre exemple. Il permet de mettre en évidence que si deux pays se touchent \og par un coin \fg{}, alors cela ne constitue pas une frontière commune.
  
 \myfigure{0.7}{
  \tikzinput{graphe-ex1-3}
}  

  
\begin{enumerate}
	  
  \item Trace le graphe associé aux cartes suivantes :
  
 \myfigure{0.8}{     
    \tikzinput{graphe-ex1-8a}\qquad
    \tikzinput{graphe-ex1-8b}\qquad
    \tikzinput{graphe-ex1-8c}
 }   
  \item Pour chacun des graphes suivants, dessine une carte qui correspond.
  
 \myfigure{1.3}{     
    \tikzinput{graphe-ex1-7c}\qquad
    \tikzinput{graphe-ex1-7a}\qquad
    \tikzinput{graphe-ex1-7b}
 }    
 
  \item Voici l'énoncé du théorème des 4 couleurs : 
  \begin{quote}
  \emph{Il est toujours possible de colorier une carte avec seulement 4 couleurs différentes, de sorte que deux pays ayant une frontière commune ne soient pas coloriés de la même couleur.}
  \end{quote}
  
  Voici un exemple sur la figure de gauche. 

  Par contre à droite, le coloriage a mal débuté, il ne sera pas possible de le terminer avec seulement 4 couleurs !
  
 \myfigure{1.3}{
  \tikzinput{graphe-ex1-4}
}    
  
  Colorie les cartes suivantes en utilisant seulement 4 couleurs et en respectant la règle de ne pas colorier deux pays voisins de la même couleur.
  
 \myfigure{1.85}{
  \tikzinput{graphe-ex1-5a}\qquad \qquad 
    \tikzinput{graphe-ex1-5b} 
}
 \myfigure{1.35}{     
    \tikzinput{graphe-ex1-5c}\qquad 
      \tikzinput{graphe-ex1-5d}
 }     

 ~
 
 \myfigure{1.3}{     
        \tikzinput{graphe-ex1-5e}
} 
  
  
  \item En termes de graphes, le théorème des 4 couleurs s'énonce ainsi : 
  \begin{quote}
  \emph{Il est toujours possible de colorier les sommets d'un graphe du plan avec seulement 4 couleurs différentes, de sorte que deux sommets reliés par une arête ne soient pas coloriés de la même couleur.}
  \end{quote}    
 
  Colorie les sommets des graphes suivants en utilisant seulement 4 couleurs.
  
 \myfigure{1.9}{
  \tikzinput{graphe-ex1-6a}
    \tikzinput{graphe-ex1-6b} 
}  
 \myfigure{1.6}{
  \tikzinput{graphe-ex1-6c}
}   
  \item Construis un exemple (simple) de carte et de graphe qui ne peut pas être colorié avec seulement 3 couleurs. Prouve ton affirmation.
\end{enumerate}
    

   
\end{activite}


\begin{activite}[Parcours d'un graphe]
On considère des villes reliées par des routes. Chaque ville est représentée par 
un sommet. Si une route relie deux villes, alors on trace une arête entre les deux sommets. De plus, sur cette arête, on écrit la distance entre les deux villes (en km).

Voici par exemple le graphe de quelques villes de France :

 \myfigure{0.9}{
  \tikzinput{graphe-ex2-1}
} 



La distance entre Lille et Paris est de 200 km.
Si je veux aller de Lille à Marseille, je peux suivre la route :
Lille - Paris - Lyon - Marseille pour un total de 1000 km, 
je peux aussi passer par Strasbourg (Lille - Strasbourg - Lyon - Marseille) pour un trajet  plus long, avec un total de 1100~km. Je pourrais aussi trouver des trajets encore plus longs en passant par Nantes...
(Note que ce n'est pas la longueur de l'arête 
qui donne la distance, mais bien le nombre associé à l'arête.)

\begin{enumerate}
  \item	\sauteligne
  
    \myfigure{1.25}{
  \tikzinput{graphe-ex2-2}
}  

  \begin{enumerate}
    \item Trouve deux chemins qui vont de $A$ à $G$ pour une somme des distances inférieure à $25$~km. 
    
    \item Il existe un chemin  de $A$ à $G$ de seulement 22 km. Peux-tu le trouver ?
  
    \item Trouve un chemin qui part de $A$ et revient à $A$ et qui passe une unique fois par toutes les arêtes (mais qui peut passer plusieurs fois par le même sommet).
  \end{enumerate}    

  \item \sauteligne
  
\myfigure{1.0}{
  \tikzinput{graphe-ex2-3}
}    

  \begin{enumerate}
    \item Trouve deux chemins qui vont de $A$ à $J$ pour une somme des distances inférieure à 34~km. 
    
    \item Il existe un chemin  de $A$ à $J$ de seulement 31 km. Peux-tu le trouver ?
  
    \item Trouve un chemin qui part de $D$ et arrive à $F$ et qui passe une unique fois par toutes les arêtes (mais qui peut passer plusieurs fois par le même sommet).  
%et qui parcourt toutes les arêtes, en passant une seule fois par chaque arête. 
  \end{enumerate} 
  

 
 
  \item Montre que pour le graphe suivant, il n'est pas possible de trouver un chemin qui part de $A$, revient à $A$, et qui parcourt une unique fois toutes les arêtes. Pour t'aider à le justifier, compte le nombre d'arêtes qui partent de chaque sommet.

 \myfigure{1.1}{
  \tikzinput{graphe-ex2-4}
}   

\end{enumerate}
      
   
\end{activite}


\begin{activite}[Caractéristique d'Euler]
Pour un graphe du plan, on compte le nombre de sommets ($S$), le nombre d'arêtes ($A$) ainsi que le nombre de faces ($F$). Les \emph{faces} sont les parties du plan à l'intérieur du graphe.

Par exemple, pour ce graphe il y a $S=4$ sommets, $A=5$ arêtes et $F=2$ faces.


    
\myfigure{1.3}{
  \tikzinput{graphe-ex3-1}
} 

 
\begin{enumerate}
  \item	Pour chacun des graphes suivants calcule $A$, $S$ et $F$.
      
\myfigure{1}{
  \tikzinput{graphe-ex3-2}
}   
  
  \item La formule d'Euler est une relation qui affirme que $S-A+F$ a toujours la même valeur, quel que soit le graphe du plan. Trouve cette valeur en te basant sur les exemples précédents :
  
 {\large 
  \mybox{$S\ -\ A\ +\ F \quad = \qquad$}
} 
\end{enumerate}
    

\end{activite}


\begin{activite}[Arbre binaire]
Un homme a $0$, $1$ ou $2$ enfants. Ses enfants (s'il en a) ont eux aussi $0$, $1$ ou $2$ enfants.
On représente la situation par un graphe, comme un arbre généalogique.

Par exemple de gauche à droite : nous avons l'homme sans enfant, l'homme avec un seul enfant et deux petits-enfants puis l'homme avec deux enfants (le premier (à gauche) sans enfant, le second (à droite) avec deux enfants).

\myfigure{0.8}{
  \small\tikzinput{graphe-ex4-1}
}     
   

\begin{enumerate}
  \item	Représente toutes les situations possibles pour trois générations.
  
  \item L'homme ayant $2$ enfants a-t-il plus de chance d'avoir $0$, $1$, $2$, $3$ ou $4$ petits-enfants ? (On supposera que toutes les situations se produisent avec la même probabilité.)
  
\end{enumerate}
   
   
\end{activite}

\end{document}


