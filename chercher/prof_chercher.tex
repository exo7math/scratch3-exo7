\documentclass[class=report,crop=false, 12pt]{standalone}
\usepackage[screen]{../myscratch}

\begin{document}

\titre[P]{Chercher et remplacer}
%===============================



\section*{Objectifs}

\begin{itemize}
  \item Activité sur texte.
\end{itemize}


\section*{Durée}

1 heure (??)

\section*{Les activités}

\begin{itemize}
  \item Il est assez fréquent que l'on veuille modifier un texte, par exemple remplacer un mot par un autre, rajouter une majuscule à "scratch"...
  
  \item Il existe bien sûr le chercher/remplacer basique.
  
  \item Il existe une version très sophistiquée, qui est abordée ici, ce sont les \emph{expressions rationnelles}, abrégées en \emph{regexp}. Cela permet de chercher (puis remplacer) à peu près n'importe quelle séquence de lettres.
  
  \item On aborde ici :
  \begin{itemize}
    \item le joker, noté "?" dans les activités, mais il est normalement noté par ".",
    \item le groupe de lettres, noté "[abc]".
    \item la négation, notée ici "!a", mais normalement notée "[\^{}a]".
  \end{itemize}
  
  \item Tout ceci permet de faire de la logique (ou, et, non) sans calcul.
  
  \item Tous les éditeurs de texte modernes autorisent les expressions rationnelles, et cela peut être une très bonne activité à faire en pratique.
\end{itemize}


\section*{Ressources}


\section*{People}

\begin{itemize}
  \item Auteur : Arnaud Bodin
\end{itemize}


\end{document}


