\documentclass[class=report,crop=false, 12pt]{standalone}
\usepackage[screen]{../myscratch}

\begin{document}

\titre[F]{Premiers pas}
%===============================



\begin{activite}[Comme une machine]
Commençons par un petit jeu : une personne doit faire tracer à d'autres un dessin bien précis.
Pour jouer, il faut :
\begin{itemize}
  \item un \emph{programmeur}, il choisit un dessin et il donne des instructions uniquement à l'oral,
  
  \item une ou plusieurs personnes qui jouent le rôle d'\emph{ordinateur}, elles doivent reproduire le dessin sans jamais l'avoir vu, juste en écoutant les instructions.
  
\end{itemize}


Le jeu se joue en trois phases, du plus facile au plus difficile. 

\bigskip

\textbf{Première phase.}

Le programmeur donne ses instructions (qu'il peut répéter). 
Les dessinateurs peuvent poser des questions, auxquelles 
le programmeur répond par oui ou non uniquement. Le programmeur peut voir ce qui est dessiné, 
mais ne peut rien montrer.

Quand tout le monde a fini son dessin, on compare avec le modèle. Puis on passe à un autre dessin.

\bigskip

\emph{Indications.} Le programmeur doit être le plus clair et le plus précis possible ! Il peut s'aider d'un quadrillage. 

\bigskip

\textbf{Deuxième phase.}

Le programmeur ne voit plus ce que font les dessinateurs. Il répond par oui ou non aux questions.

\bigskip

\textbf{Troisième phase.}

Le programmeur ne voit toujours pas ce que font les dessinateurs mais en plus il ne répond plus aux questions.

\bigskip
\bigskip

Voici quelques idées de dessins à tracer.
\myfigure{1}{
  \tikzinput{intro01}
} 

\end{activite}


\begin{activite}

Tu te déplaces sur des cases en suivant des instructions Nord, Sud, Est et Ouest.
Pour savoir quelle sera la case suivante, regardes l'instruction écrite dans la case sur laquelle tu te trouves :
\begin{itemize}
  \item si tu es sur une case \mot{N}, ta prochaine case sera celle située juste au Nord, % de la case sur laquelle tu te trouves,
  \item si tu es sur une case \mot{S}, tu te déplaceras d'une case vers le Sud,
  \item pour un case \mot{E}, tu te déplaceras vers l'Est,
  \item pour une case \mot{O}, tu te déplaceras vers l'Ouest.
\end{itemize}

\medskip

\begin{enumerate}
  \item
  \begin{enumerate}
    \item  Pars de la case A1 (en bas à gauche) et suis les instructions. Arrête-toi lorsqu'une instruction t'amène à te déplacer sur une case qui n'est pas dans la grille. Quelle sera ta position avant de quitter la grille ? (Le début du chemin est déjà tracé.)
\medskip    
    
\myfigure{0.950}{
  \tikzinput{pas01}
}     
    
    \item Repars de la case E1 sur cette nouvelle grille. Où vas-tu arriver ?  
\medskip    
      
\myfigure{0.950}{
  \tikzinput{pas02}
} 
     
  \end{enumerate} 
   
  \item
  \begin{enumerate}
    \item Pars de la case A6 et suis les instructions suivantes. Quelle sera ta case d'arrivée ?
    
    \centerline{\mot{S E S E E N E E S S S O O S}}
\medskip    
    
\myfigure{0.950}{
  \tikzinput{pas03}
}    
      
    \item Même question en partant de la case D4 avec les instructions : 
    
    \centerline{\mot{O N N E E E S S S O S O O O N}}
\medskip    
    
\myfigure{0.950}{
  \tikzinput{pas04}
}          
    
  \end{enumerate} 
 
  
  \item
  \begin{enumerate}
    \item Écris les instructions qui permettent de parcourir le chemin tracé de la case A1 à la case E6 (figure de gauche ci-dessous).
%\smallskip
\medskip    
  
\myfigure{0.950}{
  \tikzinput{pas05}
  \quad 
  \tikzinput{pas05bis}
}
     
    \item Idem pour le chemin de la case D1 à E1 (figure de droite ci-dessus). 




  \end{enumerate}   
\end{enumerate}



\end{activite}



\begin{activite}
On organise une chasse au trésor. 
On part d'une case avec une flèche et on suit un premier bloc d'instructions :
\begin{itemize}
  \item \mot{A} pour avancer d'une case (dans la direction de la flèche), 
  \item \mot{D} pour se déplacer d'une case vers la droite,
  \item \mot{G} pour se déplacer d'une case vers la gauche.
\end{itemize}
À la fin du premier un bloc d'instuctions, soit on est à nouveau sur une case contenant une flèche et on aborde un nouveau bloc d'instructions (la chasse au trésor continue), soit on a trouvé le trésor.

Voici un exemple. À l'aide de la carte, partant de la case A2 et en suivant les instructions \mot{AAG} puis \mot{AAGG}, il faut trouver l'endroit où se cache le trésor. 
\myfigure{1.50}{
\tikzinput{pas06bis}
}

\begin{itemize}
  \item On démarre de la case A2, avec une flèche qui pointe vers la droite.
  \item Premier bloc d'instructions \mot{AAG} : on avance de deux cases (dans la direction indiquée par la flèche), puis on se déplace d'une case vers la gauche (toujours par rapport à la flèche). On se retrouve donc sur la case C3.
  \item Second bloc d'instructions \mot{AAGG} : on avance de deux cases (dans la direction de la flèche de la case C3), puis deux cases vers la gauche. Le trésor se trouve donc en case E1. 
\end{itemize}

\myfigure{1.50}{
\tikzinput{pas06}
}


\begin{enumerate}
  \item On part de la case A2 et on suit les instructions :

\centerline{\mot{AAG \quad AAD \quad AD \quad AAD \quad AAG \quad AAGG \quad AAG}}  

Où est le trésor ?
  
\myfigure{1.35}{
\tikzinput{pas07}
}  
  
  \item On part de la case D4 et on suit les instructions :

\centerline{\mot{AD \quad ADD \quad AGG \quad AAGG \quad AAA \quad AAAD \quad AGG \quad AD \quad AAD}}  

Où est le trésor ?
  
\myfigure{1.35}{
\tikzinput{pas08}
}
  
  
  \item Partant de la case H3, trouve des instructions qui mènent au trésor en B5. Attention ! chaque instruction ne peut pas contenir plus de 4 lettres (par exemple \mot{AG}, \mot{AAAG}, \mot{AAGG} sont autorisées, mais pas \mot{AAAGG}).
  
\myfigure{1.35}{
\tikzinput{pas09}
}  
  \item Même question en partant de la case B3 pour atteindre le trésor en I5.
  
\myfigure{1.35}{
\tikzinput{pas10}
}  
\end{enumerate}


\end{activite}


\end{document}


