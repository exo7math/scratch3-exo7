
Vous êtes professeur de mathématiques et vous enseignez la programmation en classe ? Vous êtes parent et votre enfant s'intéresse à Scratch ? Vous êtes juste curieux de découvrir comment programmer un ordinateur ?

Ce livre "Scratch au collège" est pour vous !

Le premier objectif de ce livre, c'est d'apprendre à programmer : Scratch est le logiciel idéal pour débuter. Nous vous proposons un parcours progressif avec des énigmes ludiques pour tester vos connaissances.

Le second objectif est d'approfondir vos connaissances des algorithmes à l'aide d'activités débranchées. En travaillant sur feuille vous découvrirez des algorithmes géométriques, des algorithmes sur les mots, l'écriture binaire des nombres et plein d'autres choses !

Ce livre en noir et blanc est idéal pour vous accompagner dans votre découverte de la programmation. Il se complète par des ressources en ligne : des vidéos et des fiches en couleurs.



Activités Scratch

    Premiers pas
    Répéter
    Coordonnées x, y
    Si... alors...
    Entrée/Sortie
    Variables et hasard
    Si... alors... sinon...
    Plusieurs lutins
    Sons
    Invasion
    Créer ses blocs
    Listes


Activités débranchées

    Premiers pas
    Répéter
    Opérations algébriques I
    Vrai et faux
    Opérations algébriques II
    Si... alors...
    Boucles I
    Chercher et remplacer
    Puissances de 2
    Binaire
    Boucles II
    Graphes
    Bases de données
    Pixels
    Diviser pour régner
    Couleurs
    Cryptographie
    Triangulation
    Distance entre deux mots

