\documentclass[class=report,crop=false, 12pt]{standalone}
\usepackage{../scratch}


\begin{document}


\titre{Scratch au collège}
%=========================




\section{Objectifs}

\begin{itemize}
\item Initiation à l'informatique
\item Initiation au codage
\end{itemize}


\section{Scratch}

\begin{itemize}
\item Scratch est un langage de programmation
\item Les instructions sont des blocs à déplacer
\item Le programmation est donc visuelle, sans erreurs de syntaxe possibles
\end{itemize}

\begin{figure}[htbp]
  \centering
  \includesvg[width = 5cm]{Scratchcat}
\end{figure}


\section{Méthode}

\begin{itemize}
\item Le niveau est collège
\item Le travail est découpé en une vingtaine de séances
\item Chaque séance représente 3h de travail :
  \begin{itemize}
    \item Une heure de travail sur feuille~avec des activités non liées à
  Scratch, mais qui préparent le travail sur Scratch
    \item Une heure de travail dirigé sur machine, avec des activités sur
  Scratch, corrigées en vidéo
    \item Une heure pour résoudre des énigmes en autonomie
  \end{itemize}
\end{itemize}

%\section{Détails techniques}
%
%\begin{itemize}
%\item Chaque fiche est dans un répertoire
%\item Le travail à faire sur feuille est dans un fichier du type \emph{Feuille-02.odt}
%\item Le travail à faire sur Scratch est dans un fichier du type \emph{Scratch-02.odt}
%\item Les énigmes sont dans un fichier du type \emph{Projets-02.odt}
%\item En fait les fichiers seront écrits comme des fichiers textes, au
%  format \emph{markdown}, puis convertis au format \emph{pdf} ou
%  \emph{odt}.
%\item Les fichiers sources Scratch sont des sous-répertoires avec extension
%  \emph{.sb2}
%\item Il y a aussi des images au format \emph{jvg} ou \emph{png}
%\item Les images des blocs sont générées à partir d'un fichier texte (ex.
%  \emph{bloc-02.txt}) et transformer en image grâce au site
%  http://scratchblocks.github.io/ 
%\end{itemize}


\end{document}

