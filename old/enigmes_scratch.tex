\documentclass[11pt]{report}

\usepackage[screen]{myscratch}


\begin{document}

%%%%%%%%%%%%%%%%%%%%%%%%%%%%%%%%%%%%%%%%%%%%%%%%%%%%%%%%%%%%%%%%%%
% Titre + préface + sommaire
\renewcommand{\contentsname}{Sommaire}

% Préface
\import{divers/}{preface_scratch.tex}

\debutchapitres

%%%%%%%%%%%%%%%%%%%%%%%%%%%%%%%%%%%%%%%%%%%%%%%%%%%%%%%%%%%%%%%%%%
% Les feuilles

% 01 - Premiers pas
%\import{fiche01/}{scratch-01.tex}
\import{fiche01/}{enigme-01.tex}


% 02 - Répéter
%\import{fiche02/}{scratch-02.tex}
\import{fiche02/}{enigme-02.tex}

% 03 - Coordonnées
%\import{fiche03/}{scratch-03.tex}
\import{fiche03/}{enigme-03.tex}

% 04 - Si ... alors ...
%\import{fiche04/}{scratch-04.tex}
\import{fiche04/}{enigme-04.tex}

% 05 - Entrée/Sortie
%\import{fiche05/}{scratch-05.tex}
\import{fiche05/}{enigme-05.tex}

% 06 - Variables et hasard
%\import{fiche06/}{scratch-06.tex}
\import{fiche06/}{enigme-06.tex}

% 07 - Si ... alors ... sinon ...
%\import{fiche07/}{scratch-07.tex}
\import{fiche07/}{enigme-07.tex}

% 08 - Plusieurs lutins
%\import{fiche08/}{scratch-08.tex}
\import{fiche08/}{enigme-08.tex}

% 09 - Sons
%\import{fiche09/}{scratch-09.tex}
\import{fiche09/}{enigme-09.tex}

% 10 - Invasion
%\import{fiche10/}{scratch-10.tex}
\import{fiche10/}{enigme-10.tex}

% 11 - Créer ses blocs
%\import{fiche11/}{scratch-11.tex}
\import{fiche11/}{enigme-11.tex}

% 12 - Listes
%\import{fiche12/}{scratch-12.tex}
\import{fiche12/}{enigme-12.tex}


\newpage
\ 
\vfill



\bigskip
\bigskip
\vspace*{5cm}

\centerline{Version 0.91 -- Mai 2017}

\vspace*{1cm}

\end{document}

